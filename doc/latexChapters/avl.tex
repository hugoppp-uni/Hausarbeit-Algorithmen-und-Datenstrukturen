\subsection{Entwurf}\label{subsec:entwurf}
\paragraph{AVL-Bedingung}
Ein AVL-Baum ist ein Binärbaum, der die zusätzliche Eigenschaft besitzt, dass
der Betrag der Balance, die als die Differenz der Höhe $h$ der beiden
Teilbäume definiert ist (siehe~\ref{eq:balance}), bei jedem Knoten maximal 1 beträgt.
Diese Eigenschaft wird AVL-Bedingung genannt.

Dabei ist die Höhe analog zum regulären Binärbaum definiert.
\begin{equation}
    bal(k) = h(T_r) - h(T_l)\label{eq:balance}
\end{equation}
Durch diese Bedingung wird sichergestellt, dass der Baum zu jedem Zeitpunkt
balanciert ist und somit das in der Einleitung beschriebene Problem der
schlechten Laufzeit des Binärbaumes durch Entartung nicht auftreten kann.

\paragraph{Rebalancierung}
Nach dem Einfügen und Löschen von Elementen kann es jeweils vorkommen, dass die
Balance eines Konten -2 oder 2 beträgt.
Somit muss der AVL-Baum nach diesen Operationen die AVL-Bedingung überprüfen,
und eventuell eine Rebalancierung vornehmen.
Dabei wird zwischen insgesamt vier Fällen unterschieden, die durch die Folge
der Balancewerte definiert sind (siehe auch Abbildung~\ref{fig:AVL-Cases}):
\begin{enumerate}
    \item Left Left: 2/1
    \item Right Right: -2/-1
    \item Left Right: 2/-1
    \item Right Left: -2/1
\end{enumerate}
Bei den ersten beiden Fällen muss jeweils lediglich eine Rotation am Knoten
ausgeführt werden.
Bei den letzteren Fällen kann der Baum nicht durch eine einzige Rotation
balanciert werden, es müssen insgesamt zwei Rotationen durchgeführt werden:
Mit der ersten Rotation wird der Left Left bzw. Right Right Case
herbeigeführt, die zweite Rotation befriedigt anschließen die AVL-Bedingung.

\paragraph{Rotation}

\begin{figure}
    \centering
    %https://fr.m.wikipedia.org/wiki/Fichier:AVL_Tree_Rebalancing.svg
    %https://github.com/LambdaSchool/Data-Structures
    \includegraphics[width= 0.5\textwidth]{img/AVL_Tree_Rebalancing.png}
    \caption{Rebalancierung}
    \label{fig:AVL-Cases}
\end{figure}




\paragraph{Links Links / Rechts Rechts}
